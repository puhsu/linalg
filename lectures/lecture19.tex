\section*{Лекция 19}
\addcontentsline{toc}{section}{\protect\numberline{}Лекция 19}%
\subsection*{Двойственное (сопряжённое) векторное пространство}
\addcontentsline{toc}{subsection}{\protect\numberline{}Двойственное (сопряжённое) векторное пространство}%
$V$ --- векторное пространство. Линейная функция на $V$ --- это линейное отображение $\alpha: V \rightarrow \F$.

\begin{Def}
  $\Hom(V, \F)$ --- двойственное (сопряженное) к $V$ пространство.
\end{Def}

\begin{Designation}
  $\Hom(V, \F) = V^{*}$ --- множество всех линейных функций.
\end{Designation}

\subsubsection*{Из общей теории линейных отображений}
\begin{enumerate}
  \item $V^{*} = \Hom(V, \F) \Rightarrow V^{*}$ --- векторное пространство;
  \item Если $\e = (e_1, \ldots, e_n)$ базис $V$, то имеется изоморфизм $V^{*} \xrightarrow{\sim} \Mat_{1 \times n}(\F)$. Умножение строки на столбец --- реализация вычисления функции на векторе.
\end{enumerate}

$\alpha \in V^{*}, \alpha \mapsto (\alpha_1, \ldots, \alpha_n)$, где $\alpha_i = \alpha(e_i)$ --- коэффициенты линейной функции $\alpha$ в базисе $\e$.
\begin{Consequence}[В конечномерном случае]
  $\dim V^{*} = n = \dim V \Rightarrow V \simeq V^{*}$.
\end{Consequence}

При $i = 1, \ldots, n$ рассмотрим функцию $\varepsilon_i \in V^{*}$ соответствующую строке $(0, \ldots, 0, 1, 0, \ldots, 0) \in \Mat_{1 \times n}$. Тогда:
\[
\varepsilon_i =
\begin{cases}
  1, i = j \\
  0,\ \text{иначе}
\end{cases}
= \delta_{ij}
\]
$\varepsilon_1, \ldots, \varepsilon_n$ --- базис в $V^{*}$.

\begin{Def}
  Базис $\varepsilon_1, \ldots, \varepsilon_n$ называется базисом двойственным к базису $\e$. Он однозначно определяется условием (*).
\end{Def}

\[
\begin{pmatrix}
  \varepsilon_1 \\
  \vdots \\
  \varepsilon_n
\end{pmatrix}
(e_1, \ldots, e_n) = E
\]

\begin{Suggestion}
  Всякий базис $\varepsilon = (\varepsilon_1, \ldots, \varepsilon_n)$ пространства $V^*$ двойственен к некоторому базису пространства $V$.
\end{Suggestion}

\begin{proof}
  Возьмём произвольный базис $\e'$ пространства $V$. Пусть $\varepsilon' = (\varepsilon_1', \ldots, \varepsilon_n')$ --- двойственный к $\e'$ базис в $V^*$. Тогда:
  \begin{equation}
    \begin{pmatrix}
      \varepsilon_1 \\
      \vdots \\
      \varepsilon_n
    \end{pmatrix}
    =
    C
    \begin{pmatrix}
      \varepsilon_1' \\
      \vdots \\
      \varepsilon_n'
    \end{pmatrix}
  \end{equation}
  для некоторой невырожденной матрицы $C \in M_n$ (координаты в матрице записаны по строкам).
  Положим $\e = (e_1', \ldots, e_n')C^{-1}$ --- некий базис в $V$. Имеем:
  \begin{equation}
    \begin{pmatrix}
      \varepsilon_1 \\
      \vdots \\
      \varepsilon_n
    \end{pmatrix}
    (e_1, \ldots, e_n) =
    C
    \begin{pmatrix}
      \varepsilon_1' \\
      \vdots \\
      \varepsilon_n'
    \end{pmatrix}
    (e_1', \ldots, e_n')C^{-1} = CEC^{-1} = E
  \end{equation}

\end{proof}

\begin{Statement}
  $V = \F^n$. Множество решений некоторой ОСЛУ является подрпространством в $\F^n$.
\end{Statement}

\begin{Theorem}
  Всякое подпространство в $\F^n$ есть множество решений некоторой ОСЛУ.
\end{Theorem}
\begin{proof}
  Пусть дано подпространство $U$ в $\F^n$. Выберем в нём базис $(v_1, \ldots, v_k)$. Рассмотрим в $(\F^n)^*$ подмножество $S \defeq \{\alpha \in (\F^n)^*\ |\ \alpha(v_1) = 0, \ldots, \alpha(v_k) = 0\}$. $S$ --- подпространство в $(\F^n)^*$, $S$ --- множество решений ОСЛУ:
  \begin{equation}
    \begin{cases}
      \alpha(v_1) = 0 \\
      \vdots \\
      \alpha(v_k) = 0
    \end{cases}
  \end{equation}
  на коэффициенты $\alpha$.

  Так как $v_1, \ldots, v_k$ линейно независимы, то ранг матрицы коэффициентов равен $k \Rightarrow \dim S = n - k$. Выберем в $S$ базис $\alpha_1, \ldots, \alpha_{n - k}$ и рассмотрим ОСЛУ:
  \begin{equation}
    \begin{cases}
      \alpha_1(x) = 0 \\
      \vdots \\
      \alpha_{n - k}(x) = 0
    \end{cases}
  \end{equation}
  относительно неизвестного вектора $x \in \F^n$.

  Пусть $U' \subseteq F^n$ --- подпространство решений этой ОСЛУ. Ранг матрицы коэффициентов равен $n - k \Rightarrow \dim U' = n - n + k = k$, но $U \subseteq U'$ по построению. Так как $\dim U = \dim U' = k$, то $U = U'$.
\end{proof}

\subsection*{Билинейные функции (формы) на векторном пространстве}
\addcontentsline{toc}{subsection}{\protect\numberline{}Билинейные функции (формы) на векторном пространстве}%
\begin{Def}
  Билинейная форма (функция) на $V$ --- это отображение $\beta: V \times V \rightarrow \F$, линейное по каждому аргументу.
  \begin{enumerate}
    \item $\beta(x_1 + x_2, y) = \beta(x_1, y) + \beta(x_2, y)$
    \item $\beta(\alpha x, y) = \alpha \beta(x, y)$
    \item $\beta(x, y_1 + y_2) = \beta(x, y_1) + \beta(x, y_2)$
    \item $\beta(x, \alpha y) = \alpha \beta(x, y)$
  \end{enumerate}
\end{Def}

\begin{Examples}\
\begin{enumerate}
\item $V = \R^n,\ \beta(x, y) = \langle x, y \rangle$ --- скалярное произведение. 
\item $V = \R^2,\ \beta(x, y) = \begin{vmatrix}x_1 & y_1 \\ x_2 & y_2\end{vmatrix}$.
\item $V = C[a, b],\ \beta(f, g) = \int_a^bf(x)g(x)dx$.
\end{enumerate}
\end{Examples}

\begin{Def}
Матрицей билинейной функции в базисе $\e$ называется матрица $B = (b_{ij})$, где $b_{ij} = \beta(e_i, e_j)$. Обозначение: $B(\beta, \e)$.
\end{Def}

Пусть $x = x_1e_1 + \ldots + x_ne_n \in V$ и $y = y_1e_1 + \ldots + y_ne_n \in V$. Тогда:
\begin{gather*}
\beta(x, y) = \beta\left(\sum_{i = 1}^{n}x_ie_i, \sum_{j = 1}^{n}y_je_j\right) = \sum_{i = 1}^{n} x_i\beta\left(e_i, \sum_{j = 1}^{n}y_je_j\right) = \\
= \sum_{i = 1}^{n}x_i\sum_{j = 1}^{n}y_j\beta(e_i, e_j) = \sum_{i = 1}^{n}\sum_{j = 1}^{n}x_ib_{ij}y_j = \\
= (x_1, \ldots, x_n)B \vvector{y} \quad (*)
\end{gather*}

\begin{Suggestion}\
\begin{enumerate}
\item Всякая билинейная функция однозначно определяется своей матрицей в базисе $\e$ (и, следовательно, в любом другом базисе).
\item Для любой матрицы $B \in M_n(F)$ существует единственная билинейная функция $\beta$ такая, что $B = B(\beta, \e)$.
\end{enumerate}
\end{Suggestion}

\begin{proof}\
\begin{enumerate}
\item Уже доказано, это следует из формулы $(*)$.
\item Определим $\beta$ по формуле $(*)$. Тогда $\beta$ --- это билинейная функция на $V$ и ее матрица есть в точности $B$. Единственность следует из все той же формулы.
\end{enumerate}
\end{proof}

Пусть $\e = (e_1, \ldots, e_n)$ и $\e' = (e'_1, \ldots, e'_n)$ --- два базиса $V$, $\beta$ --- билинейная функция на $V$. Пусть также $\e' = \e C$, где $C$ --- матрица перехода, также $B(\beta, \e) = B$ и $B(\beta, \e') = B'$.

\begin{Suggestion}
$B' = C^TBC$.
\end{Suggestion}

\begin{proof}
Рассмотрим представление вектора $x \in V$ в обоих базисах.
\begin{gather*}
\begin{aligned}
x = x_1e_1 + \ldots + x_ne_n = (e_1, \ldots, e_n)\vvector{x} \\
x = x'_1e'_1 + \ldots + x'_ne'_n = (e'_1, \ldots, e'_n) \vvector{x'}
\end{aligned}
\Longrightarrow
\vvector{x} = C\vvector{x'}
\end{gather*}
Аналогично для $y \in V$:
\begin{gather*}
\begin{aligned}
y = (e_1, \ldots, e_n)\vvector{y} \\
y = (e'_1, \ldots, e'_n) \vvector{y'}
\end{aligned}
\Longrightarrow
\vvector{y} = C\vvector{y'}.
\end{gather*}
Тогда,  если мы транспонируем формулу для $x$, получаем:
$$
\beta(x, y) = \vector{x}B\vvector{y} = \vector{x'}C^TBC\vvector{y'}.
$$
Одновременно с этим:
$$
\beta(x, y) = \vector{x'}B'\vvector{y'}.
$$
Сравнивая эти две формулы, получаем, что $B' = C^TBC$.
\end{proof}
