\section*{Лекция 27}
\addcontentsline{toc}{section}{\protect\numberline{}Лекция 27}%
\subsection*{Взаимное расположение двух плоскостей в $\R^3$}
\addcontentsline{toc}{subsection}{\protect\numberline{}Взаимное расположение двух плоскостей в $\R^3$}%
\begin{enumerate}
  \item Совпадают
  \item Параллельны
  \item Пересекаются по прямой $[n_1, n_2] = 0$
\end{enumerate}
\subsection*{Взаимное расположение двух прямых в $\R^3$}
\addcontentsline{toc}{subsection}{\protect\numberline{}Взаимное расположение двух прямых в $\R^3$}%
\begin{enumerate}
  \item Совпадают
  \item Параллельны
  \item Пересекаются
  \item Скрещиваются (не лежат в одной плоскости)
\end{enumerate}
\subsection*{Взаимное расположение прямой и плоскости в $\R^3$}
\addcontentsline{toc}{subsection}{\protect\numberline{}Взаимное расположение прямой и плоскости в $\R^3$}%
\begin{enumerate}
  \item $l \subseteq P$
  \item $l || P$
  \item $l$ и $P$ пересекаются в точке
\end{enumerate}
\subsection*{Взаимное расположение трёх различных плоскостей}
\begin{enumerate}
\item Среди $P_1, P_2, P_3$ есть две параллельных
  \begin{enumerate}
    \item $P_1 \parallel P_2 \parallel P_3$
    \item Две параллельны, третья их 
  \end{enumerate}
\item Никакие две не параллельны
  \begin{enumerate}
    \item Все три пересекаются по одной прямой
    \item Прямые пересечения параллельны
    \item $P_1, P_2, P_3$ пересекаются в одной точке
  \end{enumerate}
\end{enumerate}
\subsection*{Метрические задачи в $\R^3$}
\addcontentsline{toc}{subsection}{\protect\numberline{}Метрические задачи в $\R^3$}%
\subsubsection*{Расстояние от точки до прямой}
\[
\rho(v, l) = |\ort_{\langle a \rangle}(v - v_0)| = \frac{|[v - v_0, a]|}{|a|}
\]
\subsubsection*{Расстояние от точки до плоскости}
$S = \langle n \rangle^{\perp}$ -- направляющее подпространство.
\[
\rho(v, P) = |\ort_{S}(v - v_0)| = |\pr_{\langle n \rangle}(v - v_0)| = \frac{|(v - v_0, n)|}{(n, n)}\cdot n
\]
\subsubsection*{Расстояние между скрещивающимися прямыми}
$P_1 = v_1 + \langle a_1, a_2 \rangle \supseteq l_1$, $P_2 = v_2 + \langle a_1, a_2 \rangle \supseteq l_2$
\[
\rho(l_1, l_2) = \rho(P_1, P_2) = \frac{|(a_1, a_2, v_2 - v_1)|}{|[a_1, a_2]|}
\]
\subsubsection*{Угол между двумя прямыми}
\[
\angle(l_1, l_2) = \min(\angle(a_1, a_2), \angle(a_1, -a_2))
\]
\subsubsection*{Угол между плоскостью и прямой}
\[
\angle(l, P) = \frac{\pi}{2} - \angle(\langle a \rangle, \langle n \rangle)
\]
\subsubsection*{Угол между двумя плоскостями}
\[
\angle(P_1, P_2) = \angle(\langle n_1 \rangle, \langle n_2 \rangle)
\]

\subsection*{Линейные операторы}
\addcontentsline{toc}{subsection}{\protect\numberline{}Линейные операторы}%
Пусть $V$ --- конечномерное векторное пространство.

\begin{Def}
    Линейным оператором (или линейным преобразованием) называется всякое линейное отображение $\phi \colon V \rightarrow V$, то есть из $V$ в себя. Обозначение: $L(V) = \Hom(V, V)$, $A(\phi, \e)$ -- матрица линейного оператора $\phi$ в базисе $\e$.
\end{Def}

\par Пусть $\mathbb{e} = (e_1, \ldots, e_n)$ --- базис в $V$ и $\phi \in L(V)$. Тогда:
$$
\left(\phi(e_1), \ldots, \phi(e_n)\right) = \left(e_1, \ldots, e_n\right)A,
$$
где $A$ --- матрица линейного оператора в базисе $\mathbb{e}$. В столбце $A^{\left( j\right)}$ стоят координаты $\phi(e_j)$ в базисе $\mathbb{e}$. Матрица $A$ --- квадратная. 
\begin{Examples}\
    \begin{enumerate}
        \item Скалярный оператор $\lambda \id(v) = \lambda V$ --- матрица $\lambda E$ в любом базисе.
        \item $\forall v \in V : \phi(v) = 0$ --- нулевая матрица.
        \item Тождественный оператор: $\forall v \in V : \id(v) = v$ --- единичная матрица.
        \item $V = \R^2$, $\phi$ -- поворот на угол $\alpha$
        \item $V = \R[x]_{\leqslant n}$, $\phi: f \mapsto f'$
    \end{enumerate}
\end{Examples}

\begin{Consequence}[Следствия из общих фактов о линейных отображениях]\
    \begin{enumerate}
        \item Всякий линейный оператор однозначно определяется своей матрицей в любом фиксированном базисе.
        \item Для всякого базиса $\e$ и всякой квадратной матрицы $A$ существует, причем единственный, линейный оператор $\phi$ такой, что матрица $\phi$ в базисе $\e$ есть $A$.
        \item Пусть $\phi \in L(V)$, $A$ --- матрица $\phi$ в базисе $\mathbb{e}$. Тогда:
        \begin{gather*}
            v = x_1e_1 + \ldots + x_ne_n\\ \phi(v) = y_1e_1 + \ldots + y_n e_n \\
            \begin{pmatrix}
                y_1\\
                \vdots \\
                y_n
            \end{pmatrix} = A \begin{pmatrix}
                x_1\\
                \vdots \\
                x_n
            \end{pmatrix}
        \end{gather*}
        \item Пусть $\e' = (e_1', \ldots, e_n')$ -- другой базис, $\e' = \e\cdot C$, $A' = A(\phi, \e')$, тогда $A' = C^{-1}AC$
    \end{enumerate}
\end{Consequence}

\begin{Consequence}\
  \begin{enumerate}
    \item Величина $\det A$ не зависит от выбора базиса. Обозначение: $\det\phi$.
    \item Величина $\tr A$ не зависит от выбора базиса
  \end{enumerate}
\end{Consequence}

\begin{proof}\
  Пусть $A'$ --- матрица $\phi$ в другом базисе. Тогда получается, что:
  \begin{enumerate}
    \item 
    \begin{gather*}
        \det A' = \det \left(C^{-1}AC\right) = \det C^{-1} \det A \det C = \det A \det C \cfrac{1}{\det C} = \det A.
    \end{gather*}
    \item
      \begin{gather*}
        \tr A' = \tr (C^{-1}AC) = \tr (ACC^{-1}) = \tr A
      \end{gather*}
  \end{enumerate}
\end{proof}

\begin{Def}
    Две матрицы $A', A \in M_n(F)$ называются подобными, если существует такая матрица $C \in M_n(F), \det C \neq 0$, что $A' = C^{-1}AC$.
\end{Def}

\begin{Note}
    Отношение подобия на $M_n$ является отношением эквивалентности. 
\end{Note}

\begin{Suggestion}
    Пусть $\phi \in L(V)$. Тогда эти условия эквивалентны:
    \begin{enumerate}
        \item $\Ker\phi = \{0\}$;
        \item $\Im \phi = V$;
        \item $\phi$ обратим (то есть это биекция, изоморфизм);
        \item $\det \phi \neq 0$.
    \end{enumerate}
\end{Suggestion}

\begin{proof}\ 
    \begin{enumerate}
        \item $\Leftrightarrow$ 2 --- следует из формулы $\dim V = \dim \Ker \phi + \dim \Im \phi$.
        \item $\Leftrightarrow$ 3 --- уже было.
        \item $2 \Leftrightarrow 4$ --- $\Im \phi = V \iff \rk \phi = \dim V \iff \det \phi \ne 0$.
    \end{enumerate}
\end{proof}

\begin{Def}
    Линейный оператор $\phi$ называется вырожденным, если $\det \phi = 0$, и невырожденным, если $\det \phi \neq 0$.
\end{Def}
