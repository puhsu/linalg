\section*{Лекция 21}
\addcontentsline{toc}{section}{\protect\numberline{}Лекция 21}%
$V$ -- векторное пространство над $\mathbb{F}$ (в котором $1 + 1 \ne 0$) \\
$\e = \vector{e}$ -- базис \\
$Q: V \rightarrow \mathbb{F}$ -- квадратичная форма
\begin{Def}
  $Q$ имеет в базисе $\e$ канонический вид, если в этом базисе
  \[
    Q(x) = b_1x_1^2 + \ldots + b_nx_n^2,\ b_i \in \mathbb{F}
  \]
  (то есть матрица квадратичной формы $Q$ в этом базисе диагональна)  
\end{Def}
\subsection*{Метод Якоби}
\addcontentsline{toc}{subsection}{\protect\numberline{}Метод Якоби}%
$\e = \vector{e}$
Рассмотрим набор векторов\\
$\e' = \vector{e'}$ такой что

\[
\tag{$\star$}
\begin{aligned}
  & e_1' = e_1 \\
  & e_2' \in e_2 + \langle  e_{1} \rangle \\
  & e_3' \in e_3 + \langle  e_{1}, e_2 \rangle \\
  & \vdots \\
  & e_n' \in e_{n} + \langle  e_{1}, \ldots, e_{n - 1} \rangle
\end{aligned}
\label{eqn}
\]

Для любого $k \in (1, \ldots, n)$ имеем $(e_1', \ldots, e_k') = (e_1, \ldots, e_k) \cdot  C_k$, где
\[
C_k = \begin{pmatrix}
  1 & \star & \star & \star & \cdots & \star\\
  0 & 1 & \star & \star & \cdots & \star\\
  0 & 0 & 1 & \star & \cdots & \star \\
  \vdots & \vdots & \vdots & \ddots & \vdots & \vdots \\
  \vdots & \vdots & \vdots & \vdots & \ddots & \vdots \\
  0 & 0 & 0 & 0 & \cdots & 1
\end{pmatrix}
\in \mathrm{M}_k(\mathbb{F})\\
\]
$\det C_k = 1 \ne 0\ \Rightarrow (e_1', \ldots, e_k')$ линейно независимы $\Rightarrow$
$\langle e_1, \ldots, e_k \rangle = \langle e_1', \ldots, e_k' \rangle$. В частности
$\e'$ -- базис пространства $V$.

Пусть $Q$ -- квадратичная форма \\
$B = B(Q, \e)$ \\
$B_k = B(Q, \e)$ -- левый верхний $k \times k$ блок в $B$\\
$\sigma_k = \sigma_k(Q, \e) = \det B_k$ -- $k$-ый угловой минор матрицы $B$.

Пусть $\e'$ -- базис $V$ удовлетворяющий условию \eqref{eqn} \\
$B' = B(Q, \e')$ \\
$B_k' = B_k(Q, \e')$ \\
$\sigma_k' = \sigma_k(Q, \e')$

\begin{Lemma}
  Для любого $k \in (1, \ldots, n)$, $\sigma_k = \sigma_k'$
  \begin{proof}
    При любом $k$ имеем $B_k' = C_k^T \cdot B_k \cdot C_k \Rightarrow$ определитель $\sigma_k' = \det C_k^T \cdot B \cdot C_k = \det B_k = \sigma_k$и
  \end{proof}
\end{Lemma}
\newpage
\begin{Theorem}
  (Метод Якоби приведения квадратичной формы к каноническому виду) Предположим, что $\sigma_k \neq 0 \forall k$, тогда существует единственный базис $\e' = \vector{e'}$ в $V$, такой что
  \begin{enumerate}
  \item $\e'$ имеет вид \eqref{eqn}
    \item в этом базисе $Q$ имеет канонический вид
  \end{enumerate}
  $\displaystyle Q(x) = \sigma_k x_1'^2 + \frac{\sigma_2}{\sigma_1}x_2'^2 + \ldots + \frac{\sigma_n}{\sigma_{n - 1}}x_n'^2$ \\[10pt]
  то есть $\displaystyle B(Q, \e') = \diag(\sigma_1, \frac{\sigma_2}{\sigma_1}, \ldots, \frac{\sigma_n}{\sigma_{n - 1}})$

  \begin{proof}
    Индукция по $n$: \\
    $n = 1$ -- верно \\
    Пусть доказано для $n - 1$ докажем для $n$ \\
    Пусть векторы $e_1', \ldots, e_{n - 1}'$ уже построены
    \[
    B(Q, (e_1' \ldots, e_{n - 1}', e_n)) =
    \begin{pmatrix}
      \sigma_1 & 0 & \cdots & \cdots & 0 & \star \\
      0 & \frac{\sigma_2}{\sigma_1} & \cdots & \cdots & 0 & \vdots \\
      0 & 0 & \ddots & \cdots & 0 & \vdots \\
      \vdots & \vdots & \vdots & \ddots & \vdots & \vdots \\
      \vdots & \vdots & \vdots & \vdots & \frac{\sigma_{n - 1}}{\sigma_{n - 2}} & \star \\
      \star & \star & \star & \cdots & \star & \star
    \end{pmatrix}
    \]
    Ищем $e_n'$ в виде $e_n + \langle e_1, \ldots, e_{n - 1} \rangle = e_n + \langle e_1', \ldots, e_{n - 1}' \rangle$ то есть в виде
    $e_n' = e_n + \lambda_1e_1' + \cdots + \lambda_{n - 1}e_{n - 1}'$.

    Пусть $\beta: V \times V \rightarrow \mathbb{F}$ -- симметрическая билинейная форма, соответствующая $Q$.
    \[
    \beta(e_k', e_n') = \beta(e_k', e_n) + \lambda_1\beta(e_k', e_1') + \cdots + \lambda_{k - 1} \beta(e_k', e_{k - 1}')
    \]
    так как $\beta(e_i', e_j') = 0$ при $1 \leqslant i, j \leqslant n - 1, i \neq j$
    \[
    \beta(e_k', e_n') = \beta(e_k', e_n) + \lambda_k(e_k', e_k')
    \]
    Тогда $\beta(e_k', e_n') = 0\ \forall k = 1, \ldots, n - 1$ тогда и только тогда, когда $\displaystyle \lambda_k = -\frac{\beta(e_k', e_n)}{\beta(e_k', e_k')} = -\beta(e_k', e_n)~\cdot~\frac{\sigma_{k - 1}}{\sigma_k}$.

    В итоге построен базис $\e' = \vector{e'}$ такой что
    \[
    B(Q, \e') = 
    \begin{pmatrix}
      \sigma_1 & 0 & \cdots & \cdots & 0 & 0 \\
      0 & \frac{\sigma_2}{\sigma_1} & \cdots & \cdots & 0 & \vdots \\
      0 & 0 & \ddots & \cdots & 0 & \vdots \\
      \vdots & \vdots & \vdots & \ddots & \vdots & \vdots \\
      \vdots & \vdots & \vdots & \vdots & \frac{\sigma_{n - 1}}{\sigma_{n - 2}} & 0 \\
      0 & 0 & 0 & \cdots & 0 & ?
    \end{pmatrix}
    \]
    В силу леммы $\displaystyle \sigma_n = \sigma_n' = \sigma_1 \cdot \frac{\sigma_2}{\sigma_1} \cdot \ldots \cdot \frac{\sigma_{n - 1}}{\sigma_{n - 2}} \cdot ? = \sigma_{n - 1} \cdot ? \Rightarrow ? = \frac{\sigma_n}{\sigma_{n - 1}}$.
  \end{proof}

\end{Theorem}

\begin{Note}
  Единственность следует из явных формул на каждом шаге.
\end{Note}

\subsection*{Нормальный вид квадратичной формы над полем $\mathbb{R}$}
\addcontentsline{toc}{subsection}{\protect\numberline{}Нормальный вид квадратичной формы над $\mathbb{R}$}%

\begin{Def}
  Квадратичная форма $Q$ имеет в базисе $\e$ нормальный вид, если в этом базисе $Q(x) = b_1x_1^2 + \ldots + b_nx_n^2$, где $b_i \in \{-1, 0, 1\}$.
\end{Def}

\begin{Suggestion}
  Для любой квадратичной формы $Q$ над полем $\R$ существует базис, в котором $Q$ принимает нормальный вид.
  \begin{proof}
    Существует базис, в котором $Q$ имеет канонический вид
    \[
    Q(x) = b_1x_1^2 + \ldots + b_nx_n^2
    \]
    Делаем невырожденную замену
    \[
    x_i =
    \begin{cases}
      \frac{x_i'}{\sqrt{|b_i|}}, &b_i \neq 0 \\
      x_i', &b_i = 0
    \end{cases}
    \]
    Тогда в новых координатах $Q$ имеет вид $Q(x) = \varepsilon_1 x_1'^2 + \ldots + \varepsilon_n x_n'^2$, где $\varepsilon_i = \sgn(b_i)$.
  \end{proof}
\end{Suggestion}

\begin{Note}
  Если $F = \C$, то такое же рассуждение позволяет привести любую квадратичную форму над полем $\C$ к виду $x_1^2 + \ldots + x_k^2$, где $k = \rk Q$
\end{Note}

\subsection*{Закон инерции}
\addcontentsline{toc}{subsection}{\protect\numberline{}Закон инерции}%

Пусть $Q$ -- квадратичная форма над $\R$. Нормальный вид: $Q(x) = x_1^2 + \ldots + x_s^2 - x_{s+1}^2 - \ldots - x_{s + t}^2$. $s$ -- число <<$+$>> в нормальном виде, $t$ -- число <<$-$>> в нормальном виде.

\begin{Def}
  Число $i_+ = s$ --  положительный индекс инерции квадратичной формы $Q$.
\end{Def}

\begin{Def}
  Число $i_- = t$ --  отрицательный индекс инерции квадратичной формы $Q$.
\end{Def}

\begin{Theorem}
  Числа $i_+$ и $i_-$ не зависят от базиса в котором $Q$ принимает нормальный вид.
  \begin{proof}
    Имеем $i_+ + i_- = \rk Q$ -- инвариантная величина $\Rightarrow$ достаточно доказать инвариантность числа $i_+$.

    Пусть базис $\e$ таков, что в нем $Q(x) = x_1 + \ldots + x_s^2 - x_{s + 1}^2 - \ldots - x_{s + t}^2$ и пусть $\e'$ -- другой базис такой что в нем $Q(x) = x_1^2 + \ldots + x_{s'}^2 - x_{s' + 1}^2 - \ldots - x_{s' + t'}^2$.

    Предположим, что $s \neq s'$, тогда без ограничения общности $s > s'$. Рассмотрим в $V$ подпространства $L = \langle e_1, \ldots, e_s \rangle$ и $L' = \langle e'_{s' + 1}, \ldots, e'_n \rangle$. $\dim L = s$ и $\dim L' = n - s'$. $L + L'$ -- подпространство в $V \Rightarrow \dim(L + L') \leqslant \dim V = n$. Тогда $\dim(L \cap L') = \dim L + \dim L' - \dim(L + L') \geqslant s + n - s' - n = s - s' > 0$. Тогда существует $v \in L \cap L', v \neq 0$. Так как $v \in L$, то $Q(v) > 0$, но так как $v \in L'$, то $Q(v) \leqslant 0$ -- противоречие.
  \end{proof}
\end{Theorem}

\begin{Def}
Квадратичная функция $Q$ над полем $\R$ называется
\begin{center}
\begin{tabular}{c|c|c}
Термин                      & Обозначение     & Условие \\ \hline
положительно определенной   & $Q > 0$         & $Q(x) > 0\ \forall x \neq 0$ \\
отрицательно определенной   & $Q < 0$         & $Q(x) < 0\ \forall x \neq 0$ \\
неотрицательно определенной & $Q \geqslant 0$ & $Q(x) \geqslant 0\ \forall x$ \\
неположительно определенной & $Q \leqslant 0$ & $Q(x) \leqslant 0\ \forall x$ \\
неопределенной              & $-$             & $\exists x, y \colon Q(x) > 0,\ Q(y) < 0$
\end{tabular}

\begin{tabular}{c|c|c}
Термин                      & Нормальный вид                                                      & Индексы инерции \\ \hline
положительно определенной   & $x_1^2 + \ldots + x_n^2$                                            & $i_+=n,\ i_- = 0$ \\
отрицательно определенной   & $-x_1^2 - \ldots - x_n^2$                                           & $i_+=0,\ i_-=n$ \\
неотрицательно определенной & $x_1^2 + \ldots + x_k^2,\ k \leqslant n$                            & $i_+=k,\ i_-=0$ \\
неположительно определенной & $-x_1^2 - \ldots - x_k^2,\ k \leqslant n$                           & $i_+=0, i_-=k$ \\
неопределенной              & $x_1^2+\ldots+x_s^2-x_{s + 1}^2-\ldots-x_{s + t}^2,\ s,t\geqslant1$ & $i_+=s,\ i_-=t$
\end{tabular}
\end{center}
\end{Def}

\begin{Examples} $V = \R^2$.
\begin{enumerate}
\item $Q(x, y) = x^2 + y^2,\ Q > 0$;
\item $Q(x, y) = - x^2 - y^2,\ Q < 0$;
\item $Q(x, y) = x^2 - y^2$;
\item $Q(x, y) = x^2,\ Q \geqslant 0$;
\item $Q(x, y) = -x^2,\ Q \leqslant 0$.
\end{enumerate}
\end{Examples}
