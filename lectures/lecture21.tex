\section*{Лекция 21}
\addcontentsline{toc}{section}{\protect\numberline{}Лекция 21}%
$V$ -- векторное пространство над $\mathbb{F}$ (в котором $1 + 1 \ne 0$) \\
$\e = \vector{e}$ -- базис \\
$Q: V \rightarrow \mathbb{F}$ -- квадратичная форма
\begin{Def}
  $Q$ имеет в базисе $\e$ канонический вид, если в этом базисе
  \[
    Q(x) = b_1x_1^2 + \ldots + b_nx_n^2,\ b_i \in \mathbb{F}
  \]
  (то есть матрица квадратичной формы $Q$ в этом базисе диагональна)  
\end{Def}
\subsection*{Метод Якоби}
\addcontentsline{toc}{subsection}{\protect\numberline{}Метод Якоби}%
$\e = \vector{e}$
Рассмотрим набор векторов\\
$\e' = \vector{e'}$ такой что

\[
\tag{$\star$}
\begin{aligned}
  & e_1' = e_1 \\
  & e_2' \in e_2 + \langle  e_{1} \rangle \\
  & e_3' \in e_3 + \langle  e_{1}, e_2 \rangle \\
  & \vdots \\
  & e_n' \in e_{n} + \langle  e_{1}, \ldots, e_{n - 1} \rangle
\end{aligned}
\label{eqn}
\]

Для любого $k \in (1, \ldots, n)$ имеем $(e_1', \ldots, e_k') = (e_1, \ldots, e_k) \cdot  C_k$, где
\[
C_k = \begin{pmatrix}
  1 & \star & \star & \star & \cdots & \star\\
  0 & 1 & \star & \star & \cdots & \star\\
  0 & 0 & 1 & \star & \cdots & \star \\
  \vdots & \vdots & \vdots & \ddots & \vdots & \vdots \\
  \vdots & \vdots & \vdots & \vdots & \ddots & \vdots \\
  0 & 0 & 0 & 0 & \cdots & 1
\end{pmatrix}
\in \mathrm{M}_k(\mathbb{F})\\
\]
$\det C_k = 1 \ne 0\ \Rightarrow (e_1', \ldots, e_k')$ линейно независимы $\Rightarrow$
$\langle e_1, \ldots, e_k \rangle = \langle e_1', \ldots, e_k' \rangle$. В частности
$\e'$ -- базис пространства $V$.

Пусть $Q$ -- квадратичная форма \\
$B = B(Q, \e)$ \\
$B_k = B(Q, \e)$ -- левый верхний $k \times k$ блок в $B$\\
$\sigma_k = \sigma_k(Q, \e) = \det B_k$ -- $k$-ый угловой минор матрицы $B$.

Пусть $\e'$ -- базис $V$ удовлетворяющий условию \eqref{eqn} \\
$B' = B(Q, \e')$ \\
$B_k' = B_k(Q, \e')$ \\
$\sigma_k' = \sigma_k(Q, \e')$

\begin{Lemma}
  Для любого $k \in (1, \ldots, n)$, $\sigma_k = \sigma_k'$
  \begin{proof}
    При любом $k$ имеем $B_k' = C_k^T \cdot B_k \cdot C_k \Rightarrow$ определитель $\sigma_k' = \det C_k^T \cdot B \cdot C_k = \det B_k = \sigma_k$и
  \end{proof}
\end{Lemma}
\newpage
\begin{Theorem}
  (Метод Якоби приведения квадратичной формы к каноническому виду) Предположим, что $\sigma_k \neq 0 \forall k$, тогда существует единственный базис $\e' = \vector{e'}$ в $V$, такой что
  \begin{enumerate}
  \item $\e'$ имеет вид \eqref{eqn}
    \item в этом базисе $Q$ имеет канонический вид
  \end{enumerate}
  $\displaystyle Q(x) = \sigma_k x_1'^2 + \frac{\sigma_2}{\sigma_1}x_2'^2 + \ldots + \frac{\sigma_n}{\sigma_{n - 1}}x_n'^2$ \\[10pt]
  то есть $\displaystyle B(Q, \e') = \diag(\sigma_1, \frac{\sigma_2}{\sigma_1}, \ldots, \frac{\sigma_n}{\sigma_{n - 1}})$

  \begin{proof}
    Индукция по $n$: \\
    $n = 1$ -- верно \\
    Пусть доказано для $n - 1$ докажем для $n$ \\
    Пусть векторы $e_1', \ldots, e_{n - 1}'$ уже построены
    \[
      B(Q, (e_1' \ldots, e_{n - 1}', e_n)) = 
    \]
  \end{proof}

\end{Theorem}

