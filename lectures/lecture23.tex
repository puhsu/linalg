\section*{Лекция 23}
\addcontentsline{toc}{section}{\protect\numberline{}Лекция 23}%

\subsection*{Явная формула для ортогональной проекции вектора на подпространство}
\addcontentsline{toc}{subsection}{\protect\numberline{}Явная формула для ортогональной проекции вектора на подпространство}%

$\E$ -- Евклидово пространство, $S \subseteq \E$ -- подпространство, $S^{\perp} = \{x \in \E\ |\ (x, y) = 0\ \forall y \in S\}$.
\begin{Note}\ 
  \begin{enumerate}
    \item $v \in S \iff \pr_S v = v \iff \ort_s v = 0$
    \item $v \in S^{\perp} \iff \ort_s v = v \iff \pr_s v = 0$
  \end{enumerate}
\end{Note}
Пусть $\E = \R^n$ со стандартным скалярным произведением, $S \in \R^n$ -- подпространство, $(a_1, \ldots, a_k)$ -- базис в $S$. Образуем матрицу $A \in \mathrm{Mat}_{n \times k}(\R)$, где $A^{(i)} = a_i$.
\begin{Suggestion}
  Для всякого $v \in \E$ $\pr_S v = A(A^TA)^{-1}A^Tv$.
  \begin{proof}
    Корректность $A^TA$, заметим, что $A^TA = ((a_i, a_j)) = G(a_1, \ldots, a_k)$ -- невырожденная, так как $(a_1, \ldots, a_k)$ линейно независимы.

    $v \in \E$,
    \[
    x = \pr_S v \Rightarrow x \in S \Rightarrow x = \lambda_1a_1 + \ldots + \lambda_ka_k =
    A\begin{pmatrix}
    \lambda_1 \\
    \vdots \\
    \lambda_k \\
    \end{pmatrix} 
    \]

    \[
    y = \ort_S v \Rightarrow A^Ty = 0\ \text{так как $y \in S^{\perp}$ и все скалярные произведения $= 0$}
    \]

    \[
    A(A^TA)^{-1}A^Tv = A(A^TA)^{-1}A^T(x + y) = A(A^TA)^{-1}A^TA \begin{pmatrix} \lambda_1 \\ \vdots \\ \lambda_k \end{pmatrix} + A(A^TA)^{-1}A^Ty = A\begin{pmatrix}
    \lambda_1 \\
    \vdots \\
    \lambda_k \\
    \end{pmatrix} = x
    \]
  \end{proof}
\end{Suggestion}

\subsection*{Ортогональные и ортонормированные системы векторов}
\addcontentsline{toc}{subsection}{\protect\numberline{}Ортогональные и ортонормированные системы векторов}%

$\E$ -- евклидово пространство
\begin{Def}
  Система ненулевых векторов $v_1, \ldots, v_n \in \E$ называется ортогональной, если $(v_i, v_j) = 0 \forall i \ne j$ ($\iff G(v_1, \ldots, v_n)$ диагональна + невырождена).
\end{Def}

\begin{Def}
  Система ненулевых векторов $v_1, \ldots, v_n \in \E$ называется ортонормированной, если $(v_i, v_j) = 0 \forall i \ne j$ и $(v_i, v_i) = |v_i|^2 = 1 \forall i$ ($\iff G(v_1, \ldots, v_n) = E$).
\end{Def}

\begin{Note}
  Всякая ортогональная (в частности ортонормированная) система векторов линейно независима.
\end{Note}

\begin{Def}
  Базис $(e_1, \ldots, e_n)$ в $\E$ называется ортогональным (ортонормированным), если $e_1, \ldots, e_n$ ортогональная (ортонормированная) система.
\end{Def}

\begin{Examples}
  $\E = \R^n$ со стандартным скалярным произведением $(x, y) = x_1y_1 + \ldots + x_ny_n$. Стандартный базис:
  \[
  \begin{pmatrix}
    1 \\
    0 \\
    \vdots \\
    0
  \end{pmatrix}, 
  \begin{pmatrix}
    0 \\
    1 \\
    \vdots \\
    0
  \end{pmatrix}, \cdots, 
  \begin{pmatrix}
    0 \\
    0 \\
    \vdots \\
    1
  \end{pmatrix}
  \]
  является ортонормированным.
\end{Examples}

\begin{Theorem}
  Во всяком конечномерном евклидовом пространстве существует ортогональный (ортонормированный базис).
\end{Theorem}

\begin{proof}
  Уже знаем, что так как квадратичная форма $(v, v)$ положительно определена, то существует базис, в котором она принимает нормальный вид. Этот базис и есть ортонормированный.
\end{proof}

\begin{Consequence}
  Всякую ортогональную (ортонормированную) систему векторов можно дополнить до ортогонального (ортонормированного) базиса.
\end{Consequence}

\begin{proof}
  Если $(e_1, \ldots, e_k)$ такая система, то искомым дополнением будет ортогональный (соответственно ортонормированный) базис $\{e_1, \ldots, e_k\}^{\perp}$.
\end{proof}

\begin{Note}
  Если $(e_1, \ldots, e_n)$ --- ортогональный базис, то $\displaystyle \left(\frac{e_1}{|e_1|}, \ldots, \frac{e_n}{|e_n|}\right)$ --- ортонормированный базис.
\end{Note}

$\e' = (e_1', \ldots, e_n') = (e_1, \ldots, e_n)C$ --- какой-то другой базис.

\begin{Suggestion}
  $ (e_1', \ldots, e_n')$ ортонормирован $\iff C^{T}\cdot C = E$.
\end{Suggestion}
\begin{proof}
  $\e'$ ортонормирован $\iff G(e_1', \ldots, e_n') = E$, с другой стороны $G(\e') = C^TG(\e)C = C^TEC = E$, а значит $C^TC = E$.
\end{proof}

\begin{Def}
  Матрица $C \in M_n(\R)$ называется ортогональной, если $C^T C = E$.
\end{Def}

\begin{Note}
  $C^T C = E \iff C^{-1} = C^T \iff CC^T = E$.
\end{Note}

\begin{Properties}\
  \begin{enumerate}
    \item Столбцы образуют ортонормированную систему $(C^{(i)}, C^{(j)}) = \delta_{ij}$.
    \item Строки образуют ортонормированную систему $(C_{(i)}, C_{(j)}) = \delta_{ij}$. В частности $|c_{ij}| \leqslant 1 \forall i, j$.
    \item $\det C = \pm 1$
  \end{enumerate}
\end{Properties}

\begin{Examples}
  $n = 2$
  \begin{align}
    &\det C = 1, C = \begin{pmatrix}
      \cos \phi & -\sin \phi \\
      \sin \phi & \cos \phi
    \end{pmatrix} \\[10pt]
    &\det C = -1, C = \begin{pmatrix}
      \cos \phi & \sin \phi \\
      \sin \phi & -\cos \phi
    \end{pmatrix}
  \end{align}
\end{Examples}

$S \subseteq \E$ --- подпространство и $(e_1, \ldots, e_k)$ --- ортогональный базис в $S$.

\begin{Suggestion}
 $\forall v \in \E, \pr_S v = \sum \limits_{i = 1}^{k} \frac{(v, e_i)}{(e_i, e_i)}\cdot e_i$. В частности если $(e_1, \ldots, e_k)$ ортонормирован, то $ \pr_S v = \sum \limits_{i = 1}^{k} (v, e_i)\cdot e_i$
\end{Suggestion}

\begin{proof}
  $\pr_S v = \sum \limits_{i = 1}^{k} \lambda_i e_i$ применим $(\cdot, e_j)$:
  \[
  (\pr_S v, e_j) = (v, e_j) = \sum \limits_{i = 1}^{k} \lambda_i \delta_{ij} = \lambda_j(e_j, e_j)
  \]
  Первое равенство верно потому что $v = \pr_S v + \ort_S v, (v, e_j) = (\pr_S v, e_j) + (\ort_S v, e_j) = (\pr_S v, e_j)$.
\end{proof}

\subsection*{Как строить ортогональный базис?}
\addcontentsline{toc}{subsection}{\protect\numberline{}Как строить ортогональный базис?}%

$(e_1, \ldots, e_k)$ --- линейно независимая система векторов.

\subsubsection*{Метод Якоби}

$\det G(e_1, \ldots, e_k) > 0$. $i$-ый угловой минор --- это определитель $\det G(e_1, \ldots, e_i) > 0$. Следовательно метод Якоби применим, результат --- ортогональный базис $(f_1, \ldots, f_k)$.

\[
\tag{$\star$}
\begin{aligned}
  & f_1 = e_1 \\
  & f_2 \in e_2 + \langle  e_{1} \rangle \\
  & f_3 \in e_3 + \langle  e_{1}, e_2 \rangle \\
  & \vdots \\
  & f_n \in e_{n} + \langle  e_{1}, \ldots, e_{n - 1} \rangle
\end{aligned}
\label{lec23_ort_1}
\]

\begin{Suggestion}\
  $\forall\ i = 1, \ldots, k$
  \begin{enumerate}
    \item $f_i = \ort_{\langle e_1, \ldots, e_{i - 1} \rangle} e_i$;
    \item $f_i = e_i - \sum \limits_{j = 1}^{i - 1} \frac{(e_i, f_j)}{(f_j, f_j)} \cdot f_j\ (\star\star)$;
    \item $\det G(f_1, \ldots, f_i) = \det G(e_1, \ldots, e_i)$.
  \end{enumerate}
\end{Suggestion}

\begin{proof}
  Помним, что при \eqref{lec23_ort_1} $\langle e_1, \ldots, e_i \rangle = \langle f_1, \ldots, f_i \rangle \forall i$.
  \begin{enumerate}
    \item Имеем $f_i \in e_i + \langle e_1, \ldots, e_{i - 1} \rangle = e_i + \langle f_1, \ldots, f_{i - 1} \rangle$, тогда $f_i = e_i + h_i$, где $h_i \in \langle f_1, \ldots f_{i - 1} \rangle \Rightarrow e_i = f_i - h_i$, так как $f_i \in \langle f_1, \ldots, f_{i - 1} \rangle^{\perp}$, то $f_i = \ort_{\langle f_1, \ldots, f_{i - 1}\rangle }e_i = \ort_{\langle e_1, \ldots, e_{i - 1} \rangle}e_i$
    \item $f_i = \ort_{\langle f_1, \ldots, f_{i - 1} \rangle} e_i = e_i - \pr_{\langle f_1, \ldots, f_{i - 1} \rangle} e_i$, что по пред. предложению равно $e_i~-~\sum \limits_{j = 1}^{i - 1} \frac{(e_i, f_j)}{(f_j, f_j)}f_j$ 
    \item Следует из того, что $G(f_1, \ldots, f_i) = C^T G(e_1, \ldots, e_i) C$, где $C$ --- верхнетреугольная с единицами на диагонали $\Rightarrow \det C = \det C^T = 1$.
  \end{enumerate}
\end{proof}

Построение ортогонального базиса $f_1, \ldots, f_k$ по формулам $(\star\star)$ называется методом (процессом) ортогонализации Грамма-Шмидта.

\begin{Suggestion}[Теорема Пифагора]
  $x, y \in \E, x \perp y\ (x, y) = 0$, тогда $|x + y|^2 = |x|^2 + |y|^2$.
\end{Suggestion}

\begin{proof}
  $|x + y|^2 = (x + y, x + y) = (x, x) + (x, y) + (x, y) + (y, y) = |x|^2 + |y|^2$.
\end{proof}

